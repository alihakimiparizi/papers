%
% File eacl2017.tex
%
%% Based on the style files for ACL-2016
%% Based on the style files for ACL-2015, with some improvements
%%  taken from the NAACL-2016 style
%% Based on the style files for ACL-2014, which were, in turn,
%% Based on the style files for ACL-2013, which were, in turn,
%% Based on the style files for ACL-2012, which were, in turn,
%% based on the style files for ACL-2011, which were, in turn, 
%% based on the style files for ACL-2010, which were, in turn, 
%% based on the style files for ACL-IJCNLP-2009, which were, in turn,
%% based on the style files for EACL-2009 and IJCNLP-2008...

%% Based on the style files for EACL 2006 by 
%%e.agirre@ehu.es or Sergi.Balari@uab.es
%% and that of ACL 08 by Joakim Nivre and Noah Smith

\documentclass[11pt]{article}
\usepackage{eacl2017}
\usepackage{times}
\usepackage{url}
%% \usepackage{natbib}
\usepackage{latexsym}
\usepackage{xspace}
\usepackage{multirow}
\usepackage{paralist}

\eaclfinalcopy % Uncomment this line for the final submission
%\def\eaclpaperid{***} %  Enter the acl Paper ID here

%\setlength\titlebox{5cm}
% You can expand the titlebox if you need extra space
% to show all the authors. Please do not make the titlebox
% smaller than 5cm (the original size); we will check this
% in the camera-ready version and ask you to change it back.

\newcommand\BibTeX{B{\sc ib}\TeX}


\newcommand\original{\textsc{Original}\xspace}
\newcommand\twitter{\textsc{Twitter}\xspace}

\newcommand\glove{GloVe\xspace}

\newcommand{\eqnref}[1]{Equation~\ref{#1}}
\newcommand{\figref}[1]{Figure~\ref{#1}}
\newcommand{\figsref}[2]{Figures~\ref{#1} and \ref{#2}}
\newcommand{\posciteauthor}[1]{\citeauthor{#1}'s}
\newcommand{\poscite}[1]{\citeauthor{#1}'s \citeyearpar{#1}}
\newcommand{\secref}[1]{Section~\ref{#1}}
\newcommand{\secsref}[2]{Sections~\ref{#1} and \ref{#2}}
\newcommand{\sentref}[1]{(\ref{#1})}
\newcommand{\tabref}[1]{Table~\ref{#1}}
\newcommand{\tabsref}[2]{Tables~\ref{#1} and \ref{#2}}

%% \renewcommand{\ttdefault}{cmtt}

\title{Supervised and unsupervised approaches to measuring usage similarity}

% Author information can be set in various styles:
%% For several authors from the same institution:
\author{Milton King \and Paul Cook \\
Faculty of Computer Science, University of New Brunswick\\
Fredericton, NB E3B 5A3, Canada\\
\tt{milton.king@unb.ca}, \tt{paul.cook@unb.ca}}

% if the names do not fit well on one line use
%         Author 1 \\ {\bf Author 2} \\ ... \\ {\bf Author n} \\
% For authors from different institutions:
% \author{Author 1 \\ Address line \\  ... \\ Address line
%         \And  ... \And
%         Author n \\ Address line \\ ... \\ Address line}
% To start a seperate ``row'' of authors use \AND, as in
% \author{Author 1 \\ Address line \\  ... \\ Address line
%         \AND
%         Author 2 \\ Address line \\ ... \\ Address line \And
%         Author 3 \\ Address line \\ ... \\ Address line}
% If the title and author information does not fit in the area allocated,
% place \setlength\titlebox{<new height>} right after
% at the top, where <new height> can be something larger than 2.25in
%% \author{}

\date{}

\begin{document}

\maketitle

\begin{abstract}
Usage similarity (USim) is an approach to determining word meaning in
context that does not rely on a sense inventory. Instead, pairs of
usages of a target lemma are rated on a scale. In this paper we
propose unsupervised approaches to USim based on embeddings for words,
contexts, and sentences, and achieve state-of-the-art results over two
USim datasets. We further consider supervised approaches to USim, and
find that although they outperform unsupervised approaches, they are
unable to generalize to lemmas that are unseen in the training data.
\end{abstract}


%% \section{Introduction}

%% Word senses are not discrete. In many cases, for a given instance of a
%% word, multiple senses from a sense inventory are applicable, and to
%% varying degrees \cite{Erk2009b}. That a clear line cannot be drawn
%% between the various senses of a word has been observed as far back as
%% \newcite{Johnson1755}. Some have gone so far as to doubt the existence of
%% word senses \cite{Kilgarriff1997}.

%% %% Quote from Johnson's preface:
%% %% The shades of meaning sometimes pass imperceptibly into each other;

%% Sense inventories also suffer from a lack of coverage. New words
%% regularly come into usage, as do new senses for established
%% words. Furthermore, domain-specific senses are often not included in
%% general-purpose sense inventories.
%% %% Citations to back these claims up???
%% This issue of coverage is particularly relevant for social media text,
%% which contains a higher rate of out-of-vocabulary words than
%% more-conventional text types \cite{Baldwin+:2013}.

%% These issues pose problems for natural language processing tasks such
%% as word sense disambiguation and induction, which rely on, and seek to
%% induce, respectively, sense inventories. In response to this,
%% alternative approaches to word meaning have been proposed that do not
%% rely on sense inventories \cite{Erk2009b}. In particular,
%% \citeauthor{Erk2009b} introduce the task of usage similarity (USim),
%% in which two usages of a given target word are rated for their
%% similarity, without reference to a sense inventory.

%% In this paper we propose an unsupervised approach to USim based on
%% word embeddings \cite{Mikolov+:2013b} that achieves state-of-the-art
%% results over two USim datasets, based on Twitter text and
%% more-conventional texts, respectively.  We then consider a supervised
%% approach to USim. Although this method outperforms the
%% unsupervised approaches, it performs very poorly on lemmas that are
%% unseen in the training data.

%% %% Story: Apply a range of approaches to embeddings to USim. An approach
%% %% based an average of word2vec embeddings is able to beat the previous
%% %% state of the art. Context2Vec gives further improvements in some cases.



%% \section{Related work\label{sec:rw}}

%% Word sense disambiguation (WSD) is the task of selecting the
%% most-appropriate sense --- from a sense inventory --- for a token
%% instance of a word in context \cite{Navigli:2009}. This task
%% specifically requires a sense inventory, and has traditionally assumed
%% that each instance of a word can be assigned one sense.  Word sense
%% induction (WSI), on the other hand, is the task of clustering the
%% token instances of a given word by sense, in order to induce word
%% senses. Although WSI does not require a sense inventory, this task
%% typically assumes that each usage can be assigned to one cluster
%% (i.e., induced sense). Moreover, WSI evaluation has often been based
%% on a WSD task, where induced senses are mapped to gold-standard senses
%% \cite{Agirre:Soroa:2007,Manandhar+:2010}. Recent WSI systems and
%% evaluations have, however, considered graded senses and multi-sense
%% applicability \cite{Jurgens:Klapaftis:2013}.

%% \newcite{Erk2009b} showed that multiple senses are often applicable to a
%% given instance of a word, and that this issue cannot be addressed
%% simply by choosing a coarser-grained sense inventory.  As part of
%% their work they carried out an annotation task on ``usage similarity''
%% (USim), in which the similarity of the meanings of two usages of a
%% given word are rated on a five-point scale.

%% \newcite{LuiBaldwin2012} proposed the first computational approach to
%% USim. They considered approaches based on topic modelling
%% \cite{Blei2003}, under a wide range of parameter settings, and found
%% that a single topic model for all target lemmas (as opposed to one
%% topic model per target lemma) 
%% %% based on the full-document context in which each target token
%% %% occurred,
%% performed best on the dataset of \newcite{Erk2009b}. \newcite{Gella+:2013}
%% considered USim on Twitter text, noting that this model of word
%% meaning seems particularly well-suited to this text type because of
%% the prevalence of out-of-vocabulary words. \citeauthor{Gella+:2013}
%% also considered topic modelling-based approaches, achieving their best
%% results using one topic model per target word, and a document
%% expansion strategy based on medium frequency hashtags to combat the
%% data sparsity of tweets due to their relatively short length. The
%% methods of \citeauthor{LuiBaldwin2012} and \citeauthor{Gella+:2013}
%% are unsupervised; they do not rely on any gold standard USim
%% annotations.
%% %% Reinforce that these would need to be set somehow... and that it's
%% %% not entirely clear how to do that?

%% Our proposed method achieves state of the art performance on
%% \posnewcite{Erk2009b} original USim dataset, and \posnewcite{Gella+:2013}
%% Twitter USim dataset.




\section{Usage similarity}

%% Word senses are not discrete. In many cases, for a given instance of a
%% word, multiple senses from a sense inventory are applicable, and to
%% varying degrees \cite{Erk2009b}. That a clear line cannot be drawn
%% between the various senses of a word has been observed as far back as
%% \newcite{Johnson1755}. Some have gone so far as to doubt the existence of
%% word senses \cite{Kilgarriff1997}.



%% Word senses are not discrete. In many cases, for a given instance of a
%% word, multiple senses from a sense inventory are applicable, and to
%% varying degrees, and this issue cannot be addressed simply by choosing
%% a coarser-grained sense inventory \cite{Erk2009b}. That a clear line
%% cannot be drawn between the various senses of a word has been observed
%% as far back as \newcite{Johnson1755}. Some have gone so far as to doubt
%% the existence of word senses \cite{Kilgarriff1997}.

%% ***** Working here: waiting for details from Milton... *****

Word senses are not discrete. In many cases, for a given instance of a
word, multiple senses from a sense inventory are applicable, and to
varying degrees \cite{Erk2009b}. For example, consider the usage of
\emph{wait} in the following sentence taken from
\newcite{Jurgens:Klapaftis:2013}:

\begin{enumerate}
%% \item[] \emph{Newsweek's coverage says the Web beats the malls: It's
%%   easier to \textbf{find} what you want, it's often cheaper, and you
%%   don't need to find parking.}

\item \emph{And is now the time to say I can hardly \textbf{wait}
  for your impending new novel about the Alamo?}
\end{enumerate}

\noindent
Annotators judged the WordNet \cite{Fellbaum1998} senses glossed as
`stay in one place and anticipate or expect something' and `look
forward to the probable occurrence of', to have applicability ratings
of 4 out of 5, and 2 out of 5, respectively, for this usage of
\emph{wait}. Moreover, \newcite{Erk2009b} also showed that this issue
cannot be addressed simply by choosing a coarser-grained sense
inventory. That a clear line cannot be drawn between the various
senses of a word has been observed as far back as
\newcite{Johnson1755}. Some have gone so far as to doubt the existence of
word senses \cite{Kilgarriff1997}.


%% newcite{ -> newnewcite{

%% cite{ -> newcite{

%% Quote from Johnson's preface:
%% The shades of meaning sometimes pass imperceptibly into each other;

Sense inventories also suffer from a lack of coverage. New words
regularly come into usage, as do new senses for established
words. Furthermore, domain-specific senses are often not included in
general-purpose sense inventories.
%% Citations to back these claims up???
This issue of coverage is particularly relevant for social media text,
which contains a higher rate of out-of-vocabulary words than
more-conventional text types \cite{Baldwin+:2013}.

%% These issues pose problems for natural language processing tasks such
%% as word sense disambiguation and induction, which rely on, and seek to
%% induce, respectively, sense inventories, and have traditionally
%% assumed that each instance of a word can be assigned one
%% sense.\footnote{Recent word sense induction systems and evaluations
%%   have, however, considered graded senses and multi-sense
%%   applicability \citep[e.g.,][]{Jurgens:Klapaftis:2013}.} In response
%% to this, alternative approaches to word meaning have been proposed
%% that do not rely on sense inventories \cite{Erk2009b}.

%% \newcite{Erk2009b} showed that multiple senses are often applicable to a
%% given instance of a word, and that this issue cannot be addressed
%% simply by choosing a coarser-grained sense inventory.  As part of
%% their work, they carried out an annotation task on ``usage similarity''
%% (USim), in which the similarity of the meanings of two usages of a
%% given word are rated on a five-point scale.

These issues pose problems for natural language processing tasks such
as word sense disambiguation and induction, which rely on, and seek to
induce, respectively, sense inventories, and have traditionally
assumed that each instance of a word can be assigned one
sense.\footnote{Recent word sense induction systems and evaluations
  have, however, considered graded senses and multi-sense
  applicability \cite{Jurgens:Klapaftis:2013}.} In response
to this, alternative approaches to word meaning have been proposed
that do not rely on sense inventories. \newcite{Erk2009b} carried out an
annotation task on ``usage similarity'' (USim), in which the
similarity of the meanings of two usages of a given word are rated on
a five-point scale.

%% In particular,
%% \citeauthor{Erk2009b} introduce the task of usage similarity (USim),
%% in which two usages of a given target word are rated for their
%% similarity, without reference to a sense inventory.

\newcite{LuiBaldwin2012} proposed the first computational approach to
USim. They considered approaches based on topic modelling
\cite{Blei2003}, under a wide range of parameter settings, and found
that a single topic model for all target lemmas (as opposed to one
topic model per target lemma) 
%% based on the full-document context in which each target token
%% occurred,
performed best on the dataset of \newcite{Erk2009b}. \newcite{Gella+:2013}
considered USim on Twitter text, noting that this model of word
meaning seems particularly well-suited to this text type because of
the prevalence of out-of-vocabulary words. \newcite{Gella+:2013}
also considered topic modelling-based approaches, achieving their best
results using one topic model per target word, and a document
expansion strategy based on medium frequency hashtags to combat the
data sparsity of tweets due to their relatively short length. The
methods of \newcite{LuiBaldwin2012} and \newcite{Gella+:2013}
are unsupervised; they do not rely on any gold standard USim
annotations.
%% Reinforce that these would need to be set somehow... and that it's
%% not entirely clear how to do that?

In this paper we propose unsupervised approaches to USim based on
embeddings for words \cite{Mikolov+:2013b,pennington2014glove},
contexts \cite{melamud2016context2vec}, and sentences
\cite{Kiros+:2015}, and achieve state-of-the-art results over the
USim datasets of both \newcite{Erk2009b} and \newcite{Gella+:2013}. We then
consider supervised approaches to USim based on these same methods for
forming embeddings, which outperform the unsupervised approaches, but
perform poorly on lemmas that are unseen in the training data.

%% \section{A continuous vector space model for USim\label{sec:model}}
\section{USim models\label{sec:model}}

%% Represent the meaning of a word by its context. 

%% In this paper we consider both unsupervised and supervised approaches
%% for USim.

In this section we describe how we represent a target word usage in
context, and then how we use these representations in unsupervised and
supervised approaches to USim.



\subsection{Usage representation\label{sec:usagerep}}
%%DIscuss the different methods of representing a usage for the different models


%% We use four different embedding models to represent an instance of a
%% target word.

%% \begin{compactdesc}
%% \item[Word2vec] We represent a token instance of the target word in a
%%   sentence as the average of the word2vec \cite{Mikolov+:2013b}
%%   embeddings for the other words occurring in the sentence, excluding
%%   stopwords.

%% \item[\glove] \newcite{pennington2014glove} proposed \glove, a model for
%%   learning word embeddings based on word--word co-occurrences in a
%%   corpus. We represent a target word in the same way as for word2vec,
%%   i.e., as the average of the embeddings for the other (non-stopword)
%%   words in a sentence.

%% \item[Context2vec] Context2vec \cite{melamud2016context2vec} is a
%%   bidirectional long short term memory neural network that, during
%%   training, embeds the context of word token instances in the same
%%   vector space as word types. As this model explicitly embeds word
%%   contexts it seems particularly well-suited to USim.

%% \item[Skip-thoughts] \newcite{Kiros+:2015} proposed skip-thoughts, a
%%   sentence encoder that can be viewed as a sentence-level version of
%%   word2vec's skipgram model, i.e., during training, the encoding of a
%%   sentence is used to predict surrounding sentences.  As part of their
%%   evaluation, \citeauthor{Kiros+:2015} showed that skip-thoughts
%%   out-performs previous approaches to measuring sentence-level
%%   relatedness.  Although our goal is to determine the meaning of a
%%   word in context, the meaning of a sentence could be a useful proxy
%%   for this.\footnote{Inference requires only a single sentence. The
%%     model is therefore able to infer skip-thought vectors for
%%     sentences out-of-context, and for tweets that typically do not
%%     form part of a larger discourse.}
%% \end{compactdesc}

We consider four ways of representing an instance of a target word
based on embeddings for words, contexts, and sentences. For word
embeddings, we consider word2vec \cite{Mikolov+:2013b} and \glove
\cite{pennington2014glove}. In each case we represent a token
instance of the target word in a sentence as the average of the word
embeddings for the other words occurring in the sentence, excluding
stopwords.

%% Context2vec \cite{melamud2016context2vec} is a bidirectional long
%% short term memory neural network that, during training, embeds the
%% context of word token instances in the same vector space as word
%% types. As this model explicitly embeds word contexts it seems
%% particularly well-suited to USim.

Context2vec \cite{melamud2016context2vec} can be viewed as an
extension of word2vec's continuous bag-of-words (CBOW) model. In CBOW,
the context of a target word token is represented as the average of
the embeddings for words within a fixed window. In contrast,
context2vec uses a richer representation based on a bidirectional LSTM
capturing the full sentential context of a target word token. During
training, context2vec embeds the context of word token instances in
the same vector space as word types. As this model explicitly embeds
word contexts it seems particularly well-suited to USim.

\newcite{Kiros+:2015} proposed skip-thoughts, a sentence encoder that can
be viewed as a sentence-level version of word2vec's skipgram model,
i.e., during training, the encoding of a sentence is used to predict
surrounding sentences.  \newcite{Kiros+:2015} showed that
skip-thoughts out-performs previous approaches to measuring
sentence-level relatedness.  Although our goal is to determine the
meaning of a word in context, the meaning of a sentence could be a
useful proxy for this.\footnote{Inference requires only a single
  sentence, so the model can infer skip-thought vectors for sentences
  taken out-of-context, as in the USim datasets.}

%% \footnote{Inference requires only a single
%%   sentence. The model is therefore able to infer skip-thought vectors
%%   for sentences out-of-context, and for tweets that typically do not
%%   form part of a larger discourse.  }



%%MK Is description too close to the wording from the Mikolov paper? 
%% PC: Can you provide that wording (in a comment) so I can compare?
%%"we concatenate the paragraph vector with several word vec- tors from a paragraph and predict the following word in the given context." Mikolov
%% PC: We'll worry about it if we do actually get doc2vec results...

%%MK Added a sentence to this paragraph about the word embeddings used in Doc2Vec

%% \newcite{Mikolov+:2013a} proposed doc2vec, a model that embeds a document
%% of a non-predetermined length into a vector of a predetermined length
%% by training a document vector to predict a target word's embedding in
%% a document given the embeddings of the words that occur before the
%% target word in the document. The word embeddings are trained separately using a CBOW-like model.


%% We use each of the embedding models described above to obtain a
%% representation of a given usage of a target word. In the case of
%% word2vec and \glove, we represent a token instance of the target word
%% in a sentence as the normalized sum of the word embeddings for the
%% other words occurring in the sentence, excluding stopwords. 
%% The vector sum is then normalized to account for the variation in
%% the sentences lengths.
%% For context2vec, skip-thoughts and doc2vec, we feed the sentence
%% containing the target word as input to each of these embedding models,
%% where each will create an embedding of the usage. The size of the
%% embedding is dependent on the model that generated it.

\subsection{Unsupervised approach}


%% the types corresponding to the other tokens occurring in that
%% sentence,
In the unsupervised setup, we measure the similarity between two
usages of a target word as the cosine similarity between their vector
representations, obtained by one of the methods described in
\secref{sec:usagerep}. This method does not require gold standard
training data.



\subsection{Supervised approach}

We also consider a supervised approach. For a given pair of token
instances of a target word, $t_1$ and $t_2$, we first form vectors
$v_1$ and $v_2$ representing each of the two instances of the target,
using one of the approaches in \secref{sec:usagerep}.  To represent
each pair of instances, we follow the approach of \newcite{Kiros+:2015}.
We compute the componentwise product, and absolute difference, of
$v_1$ and $v_2$, and concatenate them. This gives a vector of length
$2d$ --- where $d$ is the dimensionality of the embeddings used ---
representing each pair of instances. We then train ridge regression to
learn a model to predict the similarity of unseen usage pairs.

%% \footnote{The regression model has a regularization strength of 1 and
%%   normalizes the input.}



\section{Materials and methods\label{sec:mandm}}

%% We trained four sets of word embeddings using Wor2Vec
%% \cite{Mikolov+:2013b} on two different corpora. Two sets were trained on
%% a Twitter Dump, while the other sets were trained on a Wiki dump. One
%% set of each of the corpus dependent sets were trained using the
%% Skipgram model and the other sets were trained using the CBOW
%% model. Each set consisted of models that were trained using a variety
%% of parameters for dimensions of the embeddings and the size of the
%% training window.  We then test each individual model on two different
%% usage similarity tasks. The two datasets tested that were tested on
%% are the ones that were used in \newcite{Gella+:2013} and
%% \newcite{LuiBaldwin2012}. These are further described in section 4.1. We
%% tokenized each dataset before applying running the USim test. The
%% dataset that was used in \newcite{LuiBaldwin2012} was tokenized using a
%% basic tokenizer that we developed where all leading and trailing
%% non-alphanumeric characters are dropped and the dataset that was used
%% in \newcite{Gella+:2013} was tokenized using a tokenizer, which made to
%% handle Twitter text which is explained in 4.3. We approached this task
%% by comparing a vector representation of each sentence. The vector
%% representation of a sentence is achieved through summing over all word
%% embeddings within that sentence. The sentences were first tokenized
%% then we iterated over the sentence and if the word is not in the stop
%% word list, then the embedding of the word is grabbed from the trained
%% model and added to the ongoing sum of word embeddings using component
%% wise addition. Once both sentences within a sentence pair have a
%% vector representation, we compute the cosine distance between them as
%% measurement of similarity. In preliminary experiments, we tested with
%% different parameters for the size of the window around the word whose
%% usage similarity is being calculated. This means we only summed the
%% embeddings of the words within this window from that word. In
%% preliminary experiments, we discovered that a window, that viewed all
%% words in the sentence, shown to produce better results in terms of
%% correlation with the gold standard.

\subsection{USim Datasets \label{sec:datasets}}


We evaluate our methods on two USim datasets representing two
different text types: \original, the USim dataset of \newcite{Erk2009b},
and \twitter from \newcite{Gella+:2013}. Both USim datasets contain pairs
of sentences; each sentence in each pair includes a usage of a
particular target lemma. Each sentence pair is rated on a scale of
1--5 for how similar in meaning the usages of the target words are in
the two sentences.

\original consists of sentences from \newcite{McCarthy:Navigli:2007},
which were drawn from a web corpus \cite{Sharoff2006b}. This dataset
contains 34 lemmas, including nouns, verbs, adjectives, and
adverbs. Each lemma is the target word in 10 sentences. For each
lemma, sentence pairs (SPairs) are formed based on all pairwise
comparisons, giving 45 SPairs per lemma. Annotations were
provided by three native English speakers, with the average taken as
the final gold standard similarity. In a small number of cases the
annotators were unable to judge similarity. \newcite{Erk2009b}
removed these SPairs from the dataset, resulting in a total of
1512 SPairs.

\twitter contains SPairs for ten nouns from \original. In this
case the ``sentences'' are in fact tweets. 55 SPairs are
provided for each noun. Unlike \original, the SPairs are not
formed on the basis of all pairwise comparisons amongst a smaller set
of sentences. This dataset was annotated via crowd sourcing and
carefully cleaned to remove outlier annotations.

%% Amazon Mechanical Turk,\footnote{\url{https://www.mturk.com/}}


%% Omitted detail that they applied lexical normalization using Bo's
%% dictionary to the dataset



%% \original was tokenized by splitting on whitespace, and then stripping
%% all leading and trailing non-alphanumeric characters from tokens
%% containing at least one alphanumeric character. \twitter was tokenized
%% using a tokenizer developed for tweets, based on
%% \newcite{OConnor+:2010}. This tokenizer preserves phenomena that are
%% particularly common in tweets --- such as emoticons, hashtags, and
%% urls --- as tokens.



\subsection{Evaluation\label{sec:evaluation}}
Following \newcite{LuiBaldwin2012} and \newcite{Gella+:2013} we evaluate our
systems by calculating Spearman's rank correlation coefficient between
the gold standard similarities and the predicted similarities. This
enables direct comparison of our results with those reported in these
previous studies.


%% We considered two cross-validation methodologies to evaluate our
%% supervised approaches.

%% For experiments involving our supervised approaches, we considered two
%% evaluation methodologies based on cross-validation.


We evaluate our supervised approaches using two cross-validation
methodologies. In the first case we apply 10-fold cross-validation,
randomly partitioning all SPairs for all lemmas in a given
dataset. Using this approach, the test data for a given fold consists
of SPairs for target lemmas that were seen in the training data. To
determine how well our methods generalize to unseen lemmas, we
consider a second cross-validation setup in which we partition the
SPairs in a given dataset by lemma. Here the test data for a given
fold consists of SPairs for one lemma, and the training data consists
of SPairs for all other lemmas.

%% we considered a second approach to cross-validation. In this case we
%% partitioned the sentence pairs in a given dataset by lemma. For a
%% given fold, the test data consisted of sentence pairs for one lemma,
%% and the training data consisted of sentence pairs for all other
%% lemmas.



%% \subsection{Corpora\label{sec:corpora}}

%% % Describe corpora (Twitter and Wikipedia) and their preprocessing

%% The word embeddings were trained on three corpora: (1) a sample of
%% English tweets collected from the Twitter Streaming
%% APIs\footnote{\url{https://dev.twitter.com/}} from November 2014 to
%% March 2015, consisting of roughly 1.3 billion tokens; (2) a dump of
%% English Wikipedia from 1 September 2015, containing roughly 2.6
%% billion tokens; (3) the ukWaC \newcite{Ferraresi2008} a web corpus built
%% from the \url{.uk} domain of approximately 2 billion tokens. The
%% Twitter corpus was tokenized using a tokenizer developed for tweets,
%% based on \newcite{OConnor+:2010}. This tokenizer preserves phenomena that
%% are particularly common in tweets --- such as emoticons, hashtags, and
%% URLs --- as tokens. The Wikipedia dump was first processed using
%% WP2TXT\footnote{\url{https://github.com/yohasebe/wp2txt}} to remove
%% markup. The Wikipedia corpus was then tokenized using
%% the Stanford PTBTokenizer \cite{manning-EtAl:2014:P14-5}.  

%% PC: The ukWaC comes with tokenization. (i.e. it's distributed in a
%% vertical format with one token per line) What did you do to also run
%% the Stanford tokenizer on this? (I'm guessing you concatenated ukWaC
%% tokens together with whitespace between them, and then ran the
%% Stanford tokenizer on this)

%%MK I concatenated the tokens and then split on the whitespace

%%MK Do we need to discuss what the ukWaC looks like (already
%%tokenized) or not because it is already well known?

%% PC: I'm confused. Let's discuss in person on Thursday.

%% \subsection{Word2vec settings\label{sec:w2vparams}}
%% We trained the word embeddings using word2vec's skipgram model for a
%% variety of settings of window size ($W$=1,2,5,8) and number of
%% dimensions ($D$=50,100,300) on both the Wikipedia and Twitter corpora
%% described in the previous subsection.\footnote{In preliminary
%%   experiments the alternative word2vec CBOW model achieved
%%   substantially lower correlations than skipgram, and so CBOW was not
%%   considered further. } We used a window size of 8 and 300 dimensions
%% when training on ukWaC in order to compare against context2vec, which
%% was trained on ukWaC. This window and dimension setting was chosen
%% because it performed well when trained on the Wiki and Twitter corpora
%% and applied to (\original) and (\twitter) respectively.

\subsection{Embeddings\label{sec:mm:embeddings}}

We train word2vec's skipgram model on two corpora:\footnote{In
  preliminary experiments the alternative word2vec CBOW model achieved
  substantially lower correlations than skipgram, and so CBOW was not
  considered further.} (1) a corpus of English tweets collected from
the Twitter Streaming APIs\footnote{\url{https://dev.twitter.com/}}
from November 2014 to March 2015 containing 1.3 billion tokens; and
(2) an English Wikipedia dump from 1 September 2015 containing 2.6
billion tokens. 
%% The Twitter corpus was tokenized using a tokenizer
%% based on \newcite{OConnor+:2010}.
%% %% This tokenizer preserves phenomena that are particularly common in
%% %% tweets --- such as emoticons, hashtags, and URLs --- as tokens.
%% The Wikipedia dump was cleaned using
%% WP2TXT,\footnote{\url{https://github.com/yohasebe/wp2txt}} and then
%% tokenized using the Stanford PTBTokenizer
%% \cite{manning-EtAl:2014:P14-5}. 
Because of the relatively-low cost of training word2vec, we consider
several settings of window size ($W$=2,5,8) and number of dimensions
($D$=50,100,300).  Embeddings trained on Wikipedia and Twitter are
used for experiments on \original and \twitter, respectively.




For the other embeddings we use pre-trained models. We use \glove
vectors from Wikipedia and Twitter, with 300 and 200 dimensions, for
experiments on \original and \twitter,
respectively.\footnote{\url{http://nlp.stanford.edu/projects/glove/}}
For context2vec we use a 600 dimensional model trained on the ukWaC
\cite{Ferraresi2008}, a web corpus of approximately 2 billion
tokens.\footnote{\url{https://github.com/orenmel/context2vec}} We use
a skip-thoughts model with 4800 dimensions, trained on a corpus of
books.\footnote{\url{https://github.com/ryankiros/skip-thoughts}} We
use these context2vec and skip-thoughts models for experiments on both
\original and \twitter.

%% --- a web corpus of approximately 2 billion
%% tokens.




\section{Experimental results\label{sec:results}}




%% \begin{table*}
%% \begin{center}
%% \small
%% \begin{tabular}{cc|ccc|ccc}
%% & & \multicolumn{3}{c|}{\textsc{Original}} &
%%   \multicolumn{3}{c}{\textsc{Twitter}}\\ 

%%  &  & \multirow{2}{*}{Unsupervised}  & \multicolumn{2}{c|}{Supervised} & \multirow{2}{*}{Unsupervised} & \multicolumn{2}{c}{Supervised}  \\ 
%% $W$ & $D$ & & All & Lemma & & All & Lemma\\
%% \hline
%% %%New values with the proper seed
%% 1 & 50  & 0.236* & 0.284* & 0.163 & 0.202* & 0.267 & 0.154 \\   
%% 1 & 100 & 0.256* & 0.307* & 0.142 & 0.239* & 0.287 & 0.207 \\   
%% 1 & 300 & 0.266* & 0.378* & 0.169 & 0.237* & 0.186 & 0.032 \\  
%% 2 & 50  & 0.251* & 0.310* & 0.181 & 0.231* & 0.301 & 0.128 \\   
%% 2 & 100 & 0.267* & 0.373* & 0.196 & 0.276* & 0.261\phantom{*} & 0.155\\  
%% 2 & 300 & 0.275* & 0 .421* & 0.208 & 0.266* & 0.290\phantom{*} & 0.182 \\  
%% \hline
%% 5 & 50  & 0.262* & 0.330* & 0.215 & 0.272* & 0.331 & 0.206 \\   
%% 5 & 100 & 0.273* & 0.352* & 0.226 & 0.295* & 0.278 & 0.160 \\  
%% 5 & 300 & 0 & 0 & 0 & 0.295* & 0.347* & 0.225 \\  
%% \hline
%% 8 & 50  & \textbf{0.286*} & 0.355* & 0.232 & 0.282* &0.311\phantom{*} & \textbf{0.254} \\   
%% 8 & 100 & 0.273* & 0.376* & 0.243 & 0.298* & 0.342* & 0.236 \\  
%% 8 & 300 & 0.281* & \textbf{0.434*} & 0.209 & \textbf{0.310*} & \textbf{0.379*} & 0.138 \\    
%% %% 10 & 50 & 0.279* & 0  & \textbf{0} & 0. & 0 & 0 \\  
%% %% 10 & 100& 0  &  0   & 0 & 0. & 0 & 0 \\ 
%% %% 10 & 300& 0.281*   &  \textbf{0}   & 0 & ???   & &  ??? \\
%% \hline
%% \end{tabular}
%% \caption{Spearman's $\rho$ on \original and \twitter using the
%%   unsupervised and supervised methods with Ridge Regression, with cross-validation folds
%%   based on random sampling across all lemmas (All), or holding out
%%   individual lemmas (Lemma), for word embeddings trained using a
%%   variety of settings for window size ($W$) and number of dimensions
%%   ($D$), and skip-thought vectors. The best $\rho$ for each method is
%%   shown in boldface. Significant correlations ($p<0.05$) are indicated
%%   with *. 
%% \label{tbl:results1}}

%% \end{center}
%% \end{table*}










%% \begin{table*}
%% \begin{center}
%% \small
%% \begin{tabular}{cc|ccccc|c}

%% Dataset& &Word2vec(Wiki/Twitter) & Word2vec(UkWaC) & Skip-thoughts & Context2vec & \glove & Previous state-of-the-art \\
%% \hline

%% & Unsupervised  & 0.281* &0.276* & 0.177*& \textbf{0.302*}& 0.218* & \multirow{7}{*}{0.202}\\
%% Original &All (sup)  & 0.434*& \textbf{0.450*} & 0.433* & 0.405* & 0.406*\\
%% &Abs Diff All (sup)  & 0.370* &0.429* & \textbf{0.462*} & 0.406* & 0.317*\\
%% &Prod All (sup)  & \textbf{0.386*} & 0.385* & \textbf{0.386*} & 0.352* & 0.356*\\
%% &Lemma (sup)  & \textbf{0.209} &0.237& 0.122& 0.198 & 0.228 \\
%% &Abs Diff Lemma (sup)  &  \textbf{0.186} & 0.214 &0.140 & 0.094 &0.154 \\
%% &Prod Lemma (sup)  & 0.213 & 0.237 & 0.112 & \textbf{0.214} & 0.253\\
%% \hline

%% & Unsupervised  &\textbf{0.310*} & 0& 0.095*&  0 & 0.122* &\multirow{7}{*}{0.290}  \\
%% Twitter &All (sup)  & \textbf{0.379*}& 0& 0.364*& 0& 0.319 & \\
%% &Abs Diff All (sup)  & \textbf{0.303} & 0 & 0.292* & 0 & 0.185 \\
%% &Prod All (sup)  & \textbf{0.378*} & 0 & 0.355* & 0 &0.316 \\
%% &Lemma (sup) & 0.138 & 0& 0.079 & 0& \textbf{0.149} & \\
%% &Abs Diff Lemma (sup)  & \textbf{0.172} & 0 & 0.032 & 0 & 0.113 \\
%% &Prod Lemma (sup)  & \textbf{0.213} & 0 & 0.074 & 0 & 0.155 \\
%% \hline
%% \end{tabular}
%% \caption{Spearman's $\rho$ on \original and \twitter using the
%%   unsupervised and supervised methods with Ridge Regression, with cross-validation folds
%%   based on random sampling across all lemmas (All), or holding out
%%   individual lemmas (Lemma), for word embeddings trained using a
%%   variety of settings for window size ($W$) and number of dimensions
%%   ($D$), and skip-thought vectors. The best $\rho$ for each method is
%%   shown in boldface. Significant correlations ($p<0.05$) are indicated
%%   with *. Word2vec models trained on Wiki with a window size of 8 and 300 dimensions. Context2vec was trained on UkWaC and has 600 dimensions. \glove has a dimension of 300 for \original and 200 for \twitter.
%% %%Target lemma is not included in the Word2Vec setup for above numbers 
%% \label{tbl:results2}}
%% \end{center}
%% \end{table*}



\begin{table}
\begin{center}
%% \small
\begin{tabular}{cc|ccc|ccc}
$D$ & $W$ & \textsc{Original} & \textsc{Twitter} \\
\hline
%% 50 & 1  & 0.236 & 0.202 \\
50 & 2  & 0.251 & 0.246 \\
50 & 5  & 0.262 & 0.272 \\
50 & 8  & \textbf{0.286} & 0.282 \\
\hline
%% 100 & 1 & 0.256 & 0.239 \\
100 & 2 & 0.267 & 0.248 \\
100 & 5 & 0.273 & 0.253 \\
100 & 8 & 0.273 & 0.298 \\
\hline
%% 300 & 1 & 0.266 & 0.237 \\
300 & 2 & 0.275 & 0.266 \\
300 & 5 & 0.279   & 0.295 \\
300 & 8 & 0.281 & \textbf{0.300} \\
\hline
\end{tabular}
\caption{Spearman's $\rho$ on each dataset using the unsupervised
  approach with word2vec embeddings trained using several settings for
  the number of dimensions ($D$) and window size ($W$). The best
  $\rho$ for each dataset is shown in boldface.\label{tbl:w2vresults}}
\end{center}
\end{table}



We first consider the unsupervised approach using word2vec for a
variety of window sizes and number of dimensions. Results are shown in
\tabref{tbl:w2vresults}. All correlations are significant
($p<0.05$). On both \original and \twitter, for a given number of
dimensions, as the window size is increased, $\rho$
increases. Embeddings for larger window sizes tend to better capture
semantics, whereas embeddings for smaller window sizes tend to better
reflect syntax \cite{levy2014dependency}; the more-semantic
embeddings given by larger window sizes appear to be better-suited to
the task of predicting USim. For a given window size, a higher number
of dimensions also tends to achieve higher $\rho$. For example, for a
given window size, $D=300$ gives a higher $\rho$ than $D=50$ in each
case, except for $W=8$ on \original.


%% Results for the supervised and unsupervised methods are shown in
%% \tabref{tbl:results1}. We first consider results for the unsupervised
%% approach. On both \original and \twitter, for a given window size
%% ($W$), as the number of dimensions ($D$) is increased, $\rho$
%% increases, with the exception of $W$=8 for \original, where $D$=50
%% achieves the highest correlation. Similarly, for a given number of
%% dimensions, larger window sizes achieve higher $\rho$.


%% ***** Need to account for change in results; waiting for one more
%% number from Milton *****



%% ***** Need to check text based on updated results from Milton;
%% still waiting for skip-thoughts numbers *****

\begin{table}
\begin{center}
\small
\setlength{\tabcolsep}{2pt}
\begin{tabular}{ccccc}
\multirow{2}{*}{Dataset} & \multirow{2}{*}{Embeddings} & \multirow{2}{*}{Unsupervised} & \multicolumn{2}{c}{Supervised} \\
& & & All & Lemma \\
\hline
\multirow{4}{*}{\textsc{Original}} & Word2vec & 0.281* & 0.435* & 0.220* \\
 & \glove & 0.218* & 0.410* & \textbf{0.230*} \\
 & Skip-thoughts & 0.177* & \textbf{0.436*} & 0.099* \\
 & Context2vec & \textbf{0.302*} & 0.417* & 0.172* \\
\hline
\multirow{4}{*}{\textsc{Twitter}} & Word2vec & \textbf{0.300*} & \textbf{0.384*} & \textbf{0.196*} \\
 & \glove & 0.122* & 0.314* & 0.134*\\
 & Skip-thoughts & 0.095* & 0.360* & 0.058\phantom{*} \\
 & Context2vec & 0.122* &0.193*  & 0.067\phantom{*} \\
\hline
\end{tabular}
%% \caption{Spearman's $\rho$ on \original and \twitter using the
\caption{Spearman's $\rho$ on each dataset using the unsupervised
  method, and supervised methods with cross-validation folds based on
  random sampling across all lemmas (All) and holding out individual
  lemmas (Lemma), for each embedding approach. The best $\rho$ for
  each experimental setup, on each dataset, is shown in
  boldface. Significant correlations ($p<0.05$) are indicated with
  *.\label{tab:otherresults}}
\end{center}
\end{table}



The best correlations reported by \newcite{LuiBaldwin2012} on \original,
and \newcite{Gella+:2013} on \twitter, were 0.202 and 0.29,
respectively. The best parameter settings for our unsupervised
approach using word2vec embeddings achieve higher correlations, 0.286
and 0.300, on \original and \twitter,
respectively. \newcite{LuiBaldwin2012} and \newcite{Gella+:2013}
both report drastic variation in performance for different settings of
the number of topics in their models. We also observe some variation
with respect to parameter settings; however, any of the parameter
settings considered achieves a higher correlation than
\newcite{LuiBaldwin2012} on \original. For \twitter, parameter
settings with $W \geq 5$ and $D \geq 100$ achieve a correlation
comparable to, or greater than, the best reported by
\newcite{Gella+:2013}


%% We now consider the unsupervised approach, using the other embeddings
%% discussed in \secref{sec:usagerep}. Based on the previous observation
%% for word2vec that a higher window size and number of dimensions tends
%% to perform better, we only consider the word2vec model with $W=8$ and
%% $D=300$ here. Results are shown in \tabref{tab:otherresults} in the
%% column labeled ``Unsupervised''. For \original, context2vec performs
%% best (and indeed outperforms word2vec for all parameter settings
%% considered). This result demonstrates that approaches to predicting
%% USim that explicitly embed the context of a target word can outperform
%% approaches based on averaging word embeddings (i.e., word2vec and
%% GloVe) or embedding sentences (skip-thoughts). This result is
%% particularly strong because we consider a range of parameter settings
%% for word2vec, but only used the default settings for
%% context2vec.\footnote{The context2vec model has 600 dimensions, and
%%   was trained on the ukWac, whereas our word2vec model for \original
%%   is trained on Wikipedia. To further compare these approaches we also
%%   trained word2vec on the ukWaC with 600 dimensions and a window size
%%   of 8. These word2vec settings also did not outperform context2vec.}
%% Word2vec does however perform best on \twitter. The relatively poor
%% performance of context2vec and skip-thoughts here could be due to
%% differences between the text types these embedding models were trained
%% on (a web corpus and books, respectively) and the evaluation data
%% (tweets).

We now consider the unsupervised approach, using the other
embeddings. Based on the previous findings for word2vec, we only
consider this model with $W=8$ and $D=300$ here. Results are shown in
\tabref{tab:otherresults} in the column labeled ``Unsupervised''. For
\original, context2vec performs best (and indeed outperforms word2vec
for all parameter settings considered). This result demonstrates that
approaches to predicting USim that explicitly embed the context of a
target word can outperform approaches based on averaging word
embeddings (i.e., word2vec and GloVe) or embedding sentences
(skip-thoughts). This result is particularly strong because we
consider a range of parameter settings for word2vec, but only used the
default settings for context2vec.\footnote{The context2vec model has
  600 dimensions, and was trained on the ukWac, whereas our word2vec
  model for \original is trained on Wikipedia. To further compare
  these approaches we also trained word2vec on the ukWaC with 600
  dimensions and a window size of 8. These word2vec settings also did
  not outperform context2vec.}  Word2vec does however perform best on
\twitter.  The relatively poor performance of context2vec and
skip-thoughts here could be due to differences between the text types
these embedding models were trained on and the evaluation data.
\glove performs poorly, even though it was trained on tweets for these
experiments, but that it performs less well than word2vec is
consistent with the findings for \original.


%% In an effort to explain the relatively poor performance of \glove here
%% we examined the correlation for each individual lemma 


%% PC: Is the above comment correct about the parameter settings we
%% considered for ukWaC? Or did we consider a smaller range of settings?

%%MK We did the same setting as Context2vec. Changed the last bit of
%%the footnote


%% In all cases,
%% this supervised approach is an improvement over the unsupervised
%% approach for the same parameter settings. The best correlations on
%% \original and \twitter are 0.434 and 0.379, respectively. This
%% suggests that if labeled training data is available, supervised
%% approaches can give substantial improvements over unsupervised
%% approaches to predicting USim.\footnote{These results on \original
%%   must be interpreted cautiously, however. The same sentences, albeit
%%   in different sentence pairs, occur in both the training and testing
%%   data for a given fold. This issue does not affect \twitter.}
%% However, this experimental setup does not demonstrate the extent to
%% which such a supervised approach to USim is able to generalize to
%% lemmas that are not seen during training.


%% Turning to the supervised approach, we first consider results for
%% cross-validation based on randomly partitioning all SPairs in
%% a dataset (column ``All'' in \tabref{tab:otherresults}). The best
%% correlations on \original and \twitter of 0.434 and 0.379,
%% respectively, are achieved using word2vec embeddings, although the
%% difference in performance amongst the various embedding approaches is
%% considerably less here than in the unsupervised setting. For each type
%% of embedding considered, the supervised approach is an improvement
%% over the corresponding unsupervised approach. This suggests that if
%% labeled training data is available, supervised approaches can give
%% substantial improvements over unsupervised approaches to predicting
%% USim.\footnote{These results on \original must be interpreted
%%   cautiously, however. The same sentences, albeit in different
%%   SPairs, occur in both the training and testing data for a
%%   given fold. This issue does not affect \twitter.}  However, this
%% experimental setup does not demonstrate the extent to which such a
%% supervised approach to USim is able to generalize to lemmas that are
%% not seen during training.

Turning to the supervised approach, we first consider results for
cross-validation based on randomly partitioning all SPairs in a
dataset (column ``All'' in \tabref{tab:otherresults}). The best
correlation on \twitter (0.384) is again achieved using word2vec,
while the best correlation on \original (0.434) is obtained with
skip-thoughts. The difference in performance amongst the various
embedding approaches is, however, somewhat less here than in the
unsupervised setting. For each embedding approach, and each dataset,
the correlation in the supervised setting is better than that in the
unsupervised setting, suggesting that if labeled training data is
available, supervised approaches can give substantial improvements
over unsupervised approaches to predicting USim.\footnote{These
  results on \original must be interpreted cautiously, however. The
  same sentences, albeit in different SPairs, occur in both the
  training and testing data for a given fold. This issue does not
  affect \twitter.} However, this experimental setup does not show the
extent to which the supervised approach is able to generalize to
previously-unseen lemmas.

%%MK Do we want to remove "Moreover, for each dataset, the best supervised correlation
%%is better than the best unsupervised correlation" ?. This is redundant now that we say all supervised setups outperform their unsupervised setup.

%% does show whether a supervised approach to USim is able to generalize



The column labeled ``Lemma'' in \tabref{tab:otherresults} shows
results for the supervised approach for cross-validation using
lemma-based partitioning. In these experiments, the test data consists
of usages of a target lemma that was not seen as a target lemma during
training. For each dataset, the correlations achieved here for each
type of embedding are lower than those of the corresponding
unsupervised method, with the exception of \glove. In the case of
\original, the higher correlation for \glove relative to the
unsupervised setup appears to be largely due to improved performance
on adverbs. Nevertheless, for each dataset, the correlations
achieved by \glove are still lower than those of the best unsupervised
method on that dataset. These results demonstrate that the supervised
approach generalizes poorly to new lemmas. This negative result
indicates an important direction for future work --- identifying
strategies to training supervised approaches to predicting USim that
generalize to unseen lemmas.

%% We now consider the supervised approach for cross-validation using
%% lemma-based partitioning. In these experiments, the test data consists
%% of sentence pairs for a target lemma that was not seen as a target
%% lemma during training. For each dataset, the correlations achieved
%% here for each type of embedding are lower than those of the
%% corresponding unsupervised method, with the exception of GloVe, and
%% always lower than those of the best unsupervised method on that
%% dataset. This demonstrates that the current supervised approach
%% generalizes poorly to new lemmas. This negative result indicates an
%% important direction for future work --- identifying strategies to
%% training supervised approaches to predicting USim that generalize to
%% unseen lemmas.


%% In \tabref{tbl:results2} we evaluate five different models in seven
%% different setups for each dataset. Along with the previously discussed
%% unsupervised and supervised setups, we evaluate the models in each of
%% the supervised setups --"all" and "lemma"-- but only use one of the
%% two feature sets. "Abs Diff" is the absolute difference of the two
%% usages' representations and "Prod" is the componentwise product of the
%% two usages' representations.

%% We achieved the highest correlation on \original in the unsupervised
%% setup using the embeddings produced by the context2vec model. We
%% compared these to word2vec embeddings that were trained on the same
%% corpus as context2vec --ukWaC-- and with the same number of dimensions
%% 600. Skip-thoughts was the only model that did not achieve a higher
%% correlation than \newcite{LuiBaldwin2012} on \original.

%% In terms of the supervised "All" setup for \original, the context2vec
%% model surprisingly achieves the lowest correlation of all the
%% models. Further, when we evaluate each feature set, both
%% representations that are generated by an embedding model--
%% skip-thoughts and context2vec -- performs better with only using "Abs
%% Diff" than "All", while the representation that is created through
%% summing word embeddings -- word2vec models and \glove -- perform better
%% using both feature sets. All models perform better using only "Abs
%% Diff" than only "Prod" under the "All" setup.

%% Turning to the \twitter dataset, word2vec achieved the highest
%% correlation for all setups except the lemma separated case using both
%% features. Word2vec was the only model to achieve a higher correlation
%% than \newcite{Gella+:2013} in the unsupervised setup. Skip-thoughts
%% perform poorly as expected since it was not trained on a noisy corpus
%% similar to Twitter.


 %% As part of their evaluation, \citeauthor{Kiros+:2015} showed that
 %% skip-thoughts out-performs previous approaches to sentence-level
 %% relatedness on the SICK dataset \cite{Marelli+:2014}. Although our
 %% goal is to determine the meaning of a word in context, the meaning of
 %% a sentence could be a useful proxy for this. \citeauthor{Kiros+:2015}
 %% released a pre-trained model, based on a corpus of books, which we
 %% apply to infer the skip-thought vectors for each sentence in our
 %% datasets.\footnote{Inference requires only a single sentence. The
 %%   model is therefore able to infer skip-thought vectors for sentences
 %%   out-of-context, and for tweets that typically do not form part of a
 %%   larger discourse.} We then consider these vectors in the
 %% unsupervised approach, and the supervised approach for both
 %% cross-validation settings. Results are shown in the bottom row of
 %% \tabref{tbl:results}.  Although this approach performed poorly in the
 %% unsupervised setting, it achieved the highest correlation in the
 %% ``All'' supervised setting for \original. The comparatively low
 %% correlation here on \twitter could be due to the differences in text
 %% type (i.e., books vs. tweets).  Based on this finding, in future
 %% work, we intend to evaluate alternative approaches to forming
 %% representations of sentences \citep[e.g.,][]{Le:Mikolov:2014} for the
 %% task of predicting usage similarity.



%% A number of approaches to representing the meaning of sentences have
%% recently been proposed.




%% In preliminary experiments, we tested with different parameters for
%% the size of the window around the word whose usage similarity is
%% being calculated. This means we only summed the embeddings of the
%% words within this window from that word. In preliminary
%% experiments, we discovered that a window, that viewed all words in
%% the sentence, shown to produce better results in terms of
%% correlation with the gold standard.





% Big table of results and discussion
%% \begin{table*}
%% \centering
%% \small
%% \parbox{.45\linewidth}{

%% \begin{tabular}{ c | c | c }
%% W &  Dimen & Rho\\
%% \hline		

%% 1 & 50 & 0.266 \\ 
%% 1 & 100 & 0.268 \\ 
%% 1 & 300 & 0.273 \\ 
%% 2 & 50 & 0.292 \\ 
%% 2 & 100 & 0.325 \\ 
%% 2 & 300 & 0.286 \\ 
%% 5 & 50 & 0.282 \\ 
%% 5 & 100 & 0.301 \\ 
%% 5 & 300 & 0.296 \\ 
%% 8 & 50 & 0.286 \\ 
%% 8 & 100 & 0.287 \\ 
%% 8 & 300 & 0.298 \\ 
%% 10 & 50 & 0.280 \\ 
%% 10 & 100 & 0.275 \\ 
%% 10 & 300 & Training \\ 
%%   \hline 

%% \end{tabular}
%% \caption{Results on the dataset used by \newcite{Gella+:2013}  using embeddings trained on content from Twitter. W = training Window
%% Dimen = dimension of the embeddings used. The Skipgram model was used}\label{tab:accents}
%% }
%% \parbox{.45\linewidth}{
%% \begin{tabular}{ c | c | c }

%% W &  Dimen & Rho\\	
%% \hline		

%% 1 & 50 & 0.221 \\ 
%% 1 & 100 & 0.248 \\ 
%% 1 & 300 & 0.262 \\ 
%% 2 & 50 & 0.229 \\ 
%% 2 & 100 & 0.259 \\ 
%% 2 & 300 & 0.278 \\ 
%% 5 & 50 & 0.259 \\ 
%% 5 & 100 & 0.269 \\ 
%% 5 & 300 & 0.279 \\ 
%% 8 & 50 & 0.285 \\ 
%% 8 & 100 & 0.275 \\ 
%% 8 & 300 & 0.280 \\ 
%% 10 & 50 & 0.275 \\
%% 10 & 100 & Training \\
%% 10 & 300 & 0.280 \\ 
%% \hline
%% \end{tabular}
%% \caption{Results on the dataset used by \newcite{LuiBaldwin2012} using embeddings trained on the corpus containing content from Wikipedia. W = training Window
%% Dimen = dimension of the embeddings used. The Skipgram model was used.}\label{tab:accents}
%% }
%% \parbox{.45\linewidth}{
%% \begin{tabular}{ c | c | c }
%% W &  Dimen & Rho\\
%% \hline
%% 1 & 50 & training\\
%% 1 & 100 & training\\
%% 1 & 300 & training\\
%% 2 & 50 & 0.153 \\ 
%% 2 & 100 & 0.163 \\ 
%% 2 & 300 & 0.177 \\ 
%% 5 & 50 & 0.173 \\ 
%% 5 & 100 & 0.187 \\ 
%% 5 & 300 & 0.193 \\ 
%% 8 & 50 & 0.172 \\ 
%% 8 & 100 & 0.186 \\ 
%% 8 & 300 & 0.194 \\
%% 10 & 50 & training \\
%% 10 & 100 & training \\
%% 10 & 300 & training \\
%% \hline 
%% \end{tabular}
%% \caption{Results on the dataset used by \newcite{LuiBaldwin2012} using embeddings trained on the corpus containing content from Wikipedia. W = training Window
%% Dimen = dimension of the embeddings used. The CBOW model was used.}\label{tab:accents}
%% }
%% \parbox{.45\linewidth}{
%% \begin{tabular}{ c | c | c }
%% W &  Dimen & Rho\\
%% \hline
%% 1 & 50 & Training \\
%% 1 & 100 & Training \\
%% 1 & 300 & Training \\
%% 2 & 50 & 0.126 \\ 
%% 2 & 100 & 0.169 \\ 
%% 2 & 300 & Training \\
%% 5 & 50 & Training \\
%% 5 & 100 & Training \\
%% 5 & 300 & Training \\
%% 8 & 50 & 0.165 \\ 
%% 8 & 100 & 0.189 \\ 
%% 8 & 300 & 0.200 \\
%% 10 & 50 & Training \\
%% 10 & 100 & Training \\
%% 10 & 300 & Training \\

%% \end{tabular}
%% \caption{Results on the dataset used by \newcite{LuiBaldwin2012} using embeddings trained on the corpus containing content from Twitter. W = training Window
%% Dimen = dimension of the embeddings used. The CBOW model was used.}\label{tab:accents}
%% }

%% \parbox{.45\linewidth}{
%% \begin{tabular}{ c | c | c  | c}
%% w  &  D  &  CosRho  &  Regression rhou \\
%% \hline

%% 1 & 50 & 0.221 & 0.236 \\ 
%% 1 & 100 & 0.248 & 0.291 \\ 
%% 1 & 300 & 0.262 & 0.394 \\ 
%% 2 & 50 & 0.229 & 0.305 \\ 
%% 2 & 100 & 0.259 & 0.335 \\ 
%% 2 & 300 & 0.278 & 0.42 \\ 
%% 5 & 50 & 0.259 & 0.309 \\ 
%% 5 & 100 & 0.269 & 0.355 \\ 
%% 5 & 300 & 0.279 & 0.387 \\ 
%% 8 & 50 & 0.285 & 0.325 \\ 
%% 8 & 100 & 0.275 & 0.352 \\ 
%% 8 & 300 & 0.28 & 0.407 \\ 
%% 10 & 50 & 0.275 & 0.323 \\ 
%% \end{tabular}
%% \caption{Original Dataset. Rho regression with skip thought gets 0.365}\label{tab:accents}
%% }

%% \parbox{.45\linewidth}{
%% \begin{tabular}{ c | c | c  | c}
%% w  &  D  &  CosRho  &  Regression rhou \\
%% \hline

%% 1 & 50 & 0.212 & 0.137 \\ 
%% 1 & 100 & 0.275 & 0.14 \\ 
%% 1 & 300 & 0.286 & 0.044 \\ 
%% 2 & 50 & 0.236 & 0.119 \\ 
%% 2 & 100 & 0.309 & 0.139 \\ 
%% 2 & 300 & 0.319 & 0.125 \\ 
%% 5 & 50 & 0.285 & 0.157 \\ 
%% 5 & 100 & 0.322 & 0.159 \\ 
%% 5 & 300 & 0.351 & 0.148 \\ 
%% 8 & 50 & 0.304 & 0.186 \\ 
%% 8 & 100 & 0.326 & 0.207 \\ 
%% 8 & 300 & 0.364 & 0.1 \\ 
%% 10 & 50 & 0.309 & 0.196 \\ 
%% 10 & 100 & 0.322 & 0.123 \\ 
%% \end{tabular}
%% \caption{Titter Dataset.}\label{tab:accents}
%% }
%% \end{table*}



\section{Conclusions\label{sec:conclusions}}

%% Usage similarity is an approach to determining word meaning in context
%% that does not rely on a sense inventory. For text types that have a
%% relatively high rate of out-of-vocabulary words, such as social media
%% text, many words will be missing from sense inventories. Compared to
%% approaches to determining token-level word meaning such as word-sense
%% disambiguation which rely on a sense inventory, usage similarity
%% therefore seems particularly well-suited to such text types

%% Word senses are not discrete, and multiple senses are often applicable
%% for a given usage of a word. USim is an approach to
%% determining word meaning in context that does not rely on a sense
%% inventory, and therefore addresses these concerns. For text types that
%% have a relatively-high rate of out-of-vocabulary words, such as social
%% media text, many words will be missing from sense inventories; usage
%% similarity therefore seems particularly well-suited to such text
%% types, compared to approaches such as word-sense disambiguation which
%% rely on a sense inventory.


Word senses are not discrete, and multiple senses are often applicable
for a given usage of a word. Moreover, for text types that have a
relatively-high rate of out-of-vocabulary words, such as social media
text, many words will be missing from sense inventories. USim is an
approach to determining word meaning in context that does not rely on
a sense inventory, addressing these concerns.


%% We presented an unsupervised approach to measuring USim
%% based on word embeddings that achieved state-of-the-art performance on
%% two USim datasets, one based on Twitter text, the other
%% based on more-conventional texts. We further considered a supervised
%% approach to USim, and showed that although it outperforms
%% unsupervised approaches, it is unable to generalize to lemmas that are
%% unseen in the training data.

%% We presented an unsupervised approach to measuring usage similarity
%% based on word embeddings that achieved state-of-the-art performance on
%% two usage similarity datasets, based on Twitter and more-conventional
%% text, respectively. We further considered a supervised approach to
%% usage similarity, and showed that although it outperforms unsupervised
%% approaches, it is unable to generalize to lemmas that are unseen in
%% the training data.

We proposed unsupervised approaches to USim based on embeddings for
words, contexts, and sentences. We achieved state-of-the-art results
over USim datasets based on Twitter text and more-conventional
texts. We further considered supervised approaches to USim based on
these same methods for forming embeddings, and found that although
these methods outperformed the unsupervised approaches, they performed
poorly on lemmas that were unseen in the training data. 

The approaches to learning word embeddings that we considered
(word2vec and \glove) both learn a single vector representing each
word type. There are, however, approaches that learn multiple
embeddings for each type that have been applied to predict word
similarity in context \cite[for
  example]{Huang+:2012,Neelakantan+:2014}. In future work, we intend
to also evaluate such approaches for the task of predicting usage
similarity. We also intend to consider alternative strategies to
training supervised approaches to USim in an effort to achieve better
performance on unseen lemmas.


%% Future work:
%% training strategies

%% Alternative things like skip thoughts: doc2vec, DJ's new thing

%% These are really representing the meaning of a document though. We
%% intend to look at alternative ways to form vector representations
%% of tokens.

%% Future work! For the Lemma case, for a given fold, we should
%% do some sort of cross-validation within the training data, to
%% determine the model parameters that best generalize to an unseen
%% lemma; possibly we should also include the cosine between the two
%% vectors as a feature?

%% Future work: Train a support vector regressor



%% \setlength{\baselineskip}{0.9\baselineskip}


\section*{Acknowledgments}

This research was financially supported by the Natural Sciences and
Engineering Research Council of Canada, the New Brunswick Innovation
Foundation, ACENET, and the University of New Brunswick.


%\bibliographystyle{acl_natbib}

\bibliography{../bibtex/big}
%% \bibliographystyle{acl_natbib}
\bibliographystyle{eacl2017}


\end{document}



